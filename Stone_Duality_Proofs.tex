\documentclass{article}
\usepackage[utf8]{inputenc}
\usepackage[margin=1.5in]{geometry}
\usepackage {xcolor}
\usepackage{amsthm}
\usepackage{amsmath}
\usepackage{amssymb}
\newtheorem{proposition}{Proposition}[section]
\newtheorem{theorem}[proposition]{Theorem}
\newtheorem{lemma}[proposition]{Lemma}
\newtheorem{definition}[proposition]{Definition}
\setlength\parindent{0pt}
\usepackage{hyperref}
\usepackage[capitalize]{cleveref}
\hypersetup{
    colorlinks=true,
    linkcolor=blue,
    filecolor=blue,      
    urlcolor=blue,
    pdftitle={Introduction to Stone Duality},
    pdfpagemode=FullScreen,
}

\newcommand{\meet}{\wedge}
\newcommand{\join}{\vee}
\newcommand{\bigmeet}{\bigwedge}
\newcommand{\bigjoin}{\bigvee}
% Function restriction, from https://tex.stackexchange.com/questions/22252/how-to-typeset-function-restrictions
\newcommand\restr[2]{{% we make the whole thing an ordinary symbol
  \left.\kern-\nulldelimiterspace % automatically resize the bar with \right
  #1 % the function
  \vphantom{\big|} % pretend it's a little taller at normal size
  \right|_{#2} % this is the delimiter
}}

\title{Introduction to Stone Duality}
\author{Sri Pranav Kunda}
\date{}

\begin{document}

\maketitle

These notes come from \href{https://hackmd.io/@alexhkurz/S1W8SC0Tc}{"A Short Introduction to Stone Duality"} by Dr. Alexander Kurz. Proofs of propositions are provided, and remarks on the content are added for additional background. 

% Section 1 - Structure Preserving Maps and Duality

\section{Structure Preserving Maps and Duality}

\begin{definition}
Let $f : X \to Y$. Then $f^* : 2^Y \to 2^X$ defined by, for $b \in 2^Y$,

$$f^*(b) = \{x \in X: f(x) \in b\}$$

is called the \textbf{inverse image function} of $f$.
\end{definition}

\textbf{Remark: } The inverse image function is also called the preimage and can be denoted by $f^{-1}(x)$ on other writings. 

\begin{proposition}\label{thm:inv-image-preserves-structure}
Given a function $f : X \to Y$, the inverse image function $f^*$ preserves unions, intersections, and complements. 
\end{proposition}

\begin{proof}
Let $S \subseteq 2^Y$ and suppose $x \in \bigcup_{s \in S} f^*(s)$. This means $x$ is in at least one of $f^*(s)$ so that, by definition, $f(x)$ is in at least one $s$ and $x \in f^*(\bigcup_{s \in S} s)$. \\

Since the converse follows similarly, $f^*(\bigcup_{s \in S} s) = \bigcup_{s \in S} f^*(s)$. It is easy to see that $f^*$ preserves intersections in the same way. \\

Finally, $x \in f^*(\neg a) \Longleftrightarrow f(x) \not\in a \Longleftrightarrow x \not\in f^*(a) \Longleftrightarrow x \in \neg f^*(a)$, completing the proof.
\end{proof}

\begin{definition}[Adjoints]
Let $R : 2^Y \to 2^X$. Then $L : 2^X \to 2^Y$ is \textbf{left-adjoint} to $R$ (and $R$ is \textbf{right-adjoint} to $L$) if for all $a \in 2^X$ and $b \in 2^Y$,

\begin{equation} \label{eq:ajdoints}
L(a) \subseteq b \Longleftrightarrow a \subseteq R(b).
\end{equation}
\end{definition}

\textbf{Remark:} The phrases $L$ is a left adjoint and $L$ has a right adjoint are equivalent. 

\begin{lemma}\label{lem:inv-img-adjoint-lem-1}
If $L$ is left-adjoint to $R$, $a \subseteq R(L(a))$ and $L(R(b)) \subseteq b$.
\end{lemma}

\begin{proof}
Since $L(a) \subseteq L(a)$, $a \subseteq R(L(a))$ by (\ref{eq:ajdoints}). Similar reasoning shows that $L(R(b)) \subseteq b$.
\end{proof}

\begin{definition}[Monotone Function]
A function $F : 2^X \to 2^Y$ is \textbf{monotone} if for any $a, a' \in 2^X$, $a \subseteq a' \Rightarrow F(a) \subseteq F(a')$.
\end{definition}

\begin{lemma}\label{lem:inv-image-adjoints-monotone}
If $F : 2^X \to 2^Y$ has a left or right adjoint, $F$ is monotone.
\end{lemma}

\begin{proof}
For concreteness we assume that $F$ has a right adjoint $R$. We show that both $F$ and $R$ are monotone. Define $a, a'$ as in Definition 3.\\

By \cref{lem:inv-img-adjoint-lem-1}, $a \subseteq a' \subseteq R(F(a'))$. From (\ref{eq:ajdoints}) we see that if $a \subseteq R(F(a'))$, $F(a) \subseteq F(a')$. A similar argument invoking the latter portion of \cref{lem:inv-img-adjoint-lem-1} shows that $R$ is monotone.
\end{proof}

\begin{lemma}
Let $L : 2^X \to 2^Y$ and $R : 2^Y \to 2^X$. If $L$ and $R$ are monotone and $a \subseteq R(L(a))$ and $L(R(b)) \subseteq b$ for all $a \in 2^X$ and $b \in 2^Y$, then $L$ is left-adjoint to $R$.
\end{lemma}

\begin{proof}
Suppose $L(a) \subseteq b$. Since $L$ is monotone, $a \subseteq R(L(a)) \subseteq R(b)$. Conversely, suppose $a \subseteq R(b)$. Because $R$ is monotone, $L(a) \subseteq L(R(b)) \subseteq b$. Thus $L$ is left-adjoint to $R$ (and $R$ is right-adjoint to $L$).
\end{proof}

\begin{lemma}\label{lem:inv-img-adjoint-unique}
Left and right adjoints are unique.
\end{lemma}

\begin{proof}
We omit the proof for right adjoints because it is similar to the proof for left adjoints. Let $F : 2^X \to 2^Y$ and suppose $L, L'$ are left-adjoint to $F$. Then $a \subseteq R(L'(a))$ by \cref{lem:inv-img-adjoint-lem-1}. It follows from (\ref{eq:ajdoints}) that $L(a) \subseteq L'(a)$. Likewise, $a \subseteq R(L(a)) \Rightarrow L'(a) \subseteq L(a)$. Thus $L = L'$.
\end{proof}

\begin{proposition}\label{prop:inv-image-has-adjoints}
Any function $g : 2^Y \to 2^X$ which preserves intersections and unions has a left adjoint $g_\exists : 2^X \to 2^Y$ and a right adjoint $g_\forall : 2^X \to 2^Y$. Moreover, $g_\exists$ and $g_\forall$ preserve unions and intersections respectively.
\end{proposition}

\begin{proof}

We prove more generally that (i) $g$ has a left adjoint $g_\exists$ if and only if $g$ preserves intersections and that (ii) $g$ has a right adjoint $g_\forall$ if and only if $g$ preserves unions. The proof of (ii) is excluded for brevity since it is similar to (i). \\

Define $g_\exists(a) = \bigcap \{y : a \subseteq g(y)\}$ and suppose $g$ preserves intersections. Let $a \subseteq a'$. Then $g(a \cap a') = g(a) = g(a) \cap g(a') \Rightarrow g(a) \subseteq g(a')$. Thus $g$ is monotone. Hence, $g_\exists(a) \subseteq b \Rightarrow g(b) \supseteq g(g_\exists(a)) \supseteq a$ since $g$ preserves intersections and $a \subseteq g(b) \Rightarrow g_\exists(a) \subseteq b$. It is easy to see that $g_\exists$ preserves unions from the definition, completing the proof. \\

It is useful to note that if $g$ preserves both unions and intersections, then we may write equivalently that $g_\exists(a) = \{y \in Y \mid \exists x \in g(\{y\}) : x \in a \}$. 
\end{proof}

\begin{definition}[Atom]
$a \subseteq X$ is an \textbf{atom} if for all $S \subseteq 2^X$, $a \subseteq \bigcup S$ implies that there exists an $a' \in S$ such that $a \subseteq a'$.
\end{definition}

\begin{proposition}\label{prop:powset-atoms-are-singletons}
$a \in 2^X$ is an atom if and only if there exists $x \in X$ s.t. $a = \{x\}$.
\end{proposition}

\begin{proof}
Suppose there exists $x \in X$ s.t. $a = \{x\}$. Then for any $S \subseteq 2^X$, we have that if $a \subseteq \bigcup S$, there exists some $a' \in S$ s.t. $x \in a'$. Then $a \subseteq a'$ since $a$ is a singleton containing $x$. \\ 

For the converse, suppose $a$ is an atom. First, observe that $a$ is nonempty, otherwise we may choose $S = \emptyset$ so that there exists no $a' \in S$ s.t. $a \subseteq a'$ although $a \subseteq \bigcup S$. \\

For a contradiction, suppose $|a| > 1$. Choosing $S = \{\{x\} : x \in a\}$, we see that $a \subseteq \bigcup S$. However, $a$ is not contained in any $a' \in S$. Thus $|a| = 1$ and the proof is complete.
\end{proof}

\begin{lemma}\label{lem:left-adjoint-maps-atoms-to-atoms}
Let $g : 2^Y \to 2^X$ preserve unions and intersections. Then the left adjoint of $g$ (which exists and is unique by \cref{prop:inv-image-has-adjoints} and \cref{lem:inv-img-adjoint-unique}) maps atoms to atoms. 
\end{lemma}

\begin{proof}
Let $a \in 2^X$ be an atom, let $S \subseteq 2^Y$ s.t. $g_\exists(a) \subseteq \bigcup S$, and let $T = \{g(s) : s \in S\}$. Then $a \subseteq g(\bigcup S) = \bigcup T$ by the definition of the adjoint. Since $a$ is an atom and $T \subseteq 2^X$, there exists $a' \in T$ s.t. $a \subseteq a'$. Because $a' = g(s)$ for some $s \in S$, we have that $g_\exists(a) \subseteq g_\exists(g(s)) \subseteq s$ by \cref{lem:inv-img-adjoint-lem-1,lem:inv-image-adjoints-monotone}
\end{proof}

\begin{proposition} \label{prop:powset-morphism-is-inv-image}
Every function $g : 2^Y \to 2^X$ that preserves unions and intersections is the inverse image function for a unique $g_* : X \to Y$. 
\end{proposition}

\begin{proof}
By \cref{lem:left-adjoint-maps-atoms-to-atoms}, $g$ has a unique left adjoint $g_\exists$ which maps atoms to atoms. Define $g_*(x) = g_\exists(\{x\})$. We show that $(g_*)^*$ is right adjoint to $g_\exists$ so that $g$ is the inverse image function of $g_* : X \to Y$ by the uniqueness of adjoints. \\

It has already been shown that both $g_\exists$ and $(g_*)^*$ are monotone and preserve unions. Hence, $(g_*)^*(g_\exists(a)) = \bigcup_{x \in a} (g_*)^*(g_\exists(\{x\})) \supseteq a$ and $g_\exists((g_*)^*(b)) = \bigcup_{y \in b} g_\exists((g_*)^*(\{y\})) \subseteq b$. \\

We retain the notation $g_*$ to denote the function for which $g$ is the inverse image function. 
\end{proof}

\begin{theorem} \label{thm:bijection-set-morphisms-to-powset-morphisms}
There is a bijection between functions $2^Y \to 2^X$ which preserve unions and intersections and functions $X \to Y$.
\end{theorem}

\begin{proof}
By \cref{prop:powset-morphism-is-inv-image}, there exists a unique $g_* : X \to Y$ for every union/intersection preserving $g : 2^Y \to 2^X$ so that $(g_*)^* = g$. Similarly, every $f : X \to Y$ has a unique inverse image function $f^*$ by definition. Hence we have a bijection.
\end{proof}

\pagebreak

% Section 2 - Algebraic Duality

\section{Algebraic Duality}

\subsection{Algebraic Preliminaries}

There are two equivalent definitions of a lattice, one which uses partially ordered sets and another which defines a lattice as an algebraic structure. Only the latter is below, but the former can be found on \href{https://en.wikipedia.org/wiki/Lattice_(order)}{the Wikipedia page}.

\begin{definition}[Lattice]

A \textbf{lattice} is a structure $(A, \meet, \join)$, where $A$ is a set and $\meet$ and $\join$ are binary, commutative, and associative operations on $A$ which satisfy the following (called absorption laws) for all $x, y \in A$:

\begin{enumerate}
    \item[L1.]{$x \join (x \meet y) = x$}
    \item[L2.]{$x \meet (x \join y) = x$.}
\end{enumerate}

$\meet$ is read "meet" and $\join$ is read "join."

\end{definition}

\begin{lemma}[Idempotence]
Let $(A, \meet, \join)$ be a lattice. For any $x \in A$, $$x \meet x = x \text{ and } x \join x = x.$$
\end{lemma}

\begin{proof}
Choose $y = x \join x$ from Definition 5. Then $x \meet y = x$ by (L2), meaning $$x \join (x \meet y) = x \join x = x$$ by (L1). The argument for $\meet$ follows similarly.
\end{proof}

\begin{definition}[Partial Order on Lattices]
We may define a partial order $\leq$ on a lattice $A$, which generalizes $\subseteq$ from the previous section. For $x, y \in A$, define 
\begin{align*}
    x \leq y \text{ if } x \meet y = x, & \text{ or } \\
    x \leq y \text{ if } x \join y = y.
\end{align*}
\end{definition}

\begin{definition}[Bounded Lattice]
A lattice $A$ is \textbf{bounded} (or is a bounded lattice) if there exist $0, 1 \in A$ such that $$0 \leq x \leq 1 \text{ for all } x \in A.$$ When there are multiple bounded lattices involved, we avoid confusion by using $1_A$ and $0_A$ to denote the maximum and minimum elements of the lattice $A$.
\end{definition}

\begin{definition}[Complete Lattice]
A lattice $A$ is \textbf{complete} (or is a complete lattice) if every $T \subseteq A$ has a supremum and infimum, respectively denoted $\sup T, \inf T \in A$ (see \href{https://en.wikipedia.org/wiki/Infimum_and_supremum}{infimum and supremum}).
\end{definition}

\begin{lemma}
Every complete lattice $A$ is bounded. Note that the converse is not necessarily true.
\end{lemma}

\begin{proof}
Since $A$ is complete, there exists $0_A \in A$ (resp. $1_A \in A$) such that for every lower bound (resp. upper bound) $y \in \emptyset \subseteq A,$ $1_A \geq y$ (resp. $0_A \leq y$). Since every $y \in A$ is vacuously a lower bound and an upper bound, the proof is complete.
\end{proof}

\begin{lemma}
Every finite lattice $A$ is complete. As a corollary, we have shown that all finite lattices are bounded.
\end{lemma}

\begin{proof}
The proof follows easily from the definition of a complete lattice. 
\end{proof}

\begin{lemma}
Let $A$ be a lattice. For $x, y, z \in A$, the following identities are equivalent: 
\begin{enumerate}
    \item[D1.]{$x \meet (y \join z) = (x \meet y) \join (x \meet z)$}
    \item[D2.]{$x \join (y \meet z) = (x \join y) \meet (x \join z)$}
\end{enumerate}
\end{lemma}

\begin{proof}
Assume (D1) holds. Then 
\begin{align*}
(x \join y) \meet (x \join z) &= ((x \join y) \meet x) \join ((x \join y) \meet z) 
                           \\ &= x \join ((x \join y) \meet z)
                           \\ &= (x \join (z \meet x)) \join (y \meet z)
                           \\ &= x \join (y \meet z)
\end{align*}
by associativity/commutativity of $\join$ and $\meet$. The converse follows similarly.
\end{proof}

\begin{definition}[Distributive Lattice]
A lattice $A$ is \textbf{distributive} if (D1) (or (D2), equivalently) holds for $x, y, z \in A$.
\end{definition}

\begin{definition}[Complemented Lattice]
A lattice $A$ is \textbf{complemented} if for every $x \in A$, there exists some $x' \in A$ such that $x \meet x' = 0_A$ and $x \join x' = 1_A$. We call $x'$ a complement of $x$. 
\end{definition}

\begin{definition}[Finite Boolean Algebra]
A \textbf{finite Boolean algebra} is a finite complemented distributive lattice.
\end{definition}

The following lemma allows us to speak of complements in finite Boolean algebras with reference to exactly one element. Hence, we denote the complement of $x$ by $\neg x$.

\begin{lemma} \label{lem:finite-ba-complements-are-unique}
Complements of elements in a finite Boolean algebra $C$ are unique. 
\end{lemma}

\begin{proof}
Let $x, x', x'' \in C$, where $x'$ and $x''$ are complements of $x$. Then $x' = x' \join (x \meet x'') = (x' \join x) \meet (x' \join x'') = x' \join x''$. This means $x'' \leq x'$. Interchanging $x'$ and $x''$ shows that $x' \leq x''$, completing the proof. 
\end{proof}

\begin{lemma}[De Morgan's Laws] \label{lem:de-morgans-laws}
Let $C$ be a finite Boolean algebra and let $x, y \in C$. Then $\neg (x \meet y) = \neg x \join \neg y$ and $\neg (x \join y) = \neg x \meet \neg y$.
\end{lemma}

\begin{proof}
The proof follows easily from the fact that $C$ is distributive.
\end{proof}

\begin{proposition} \label{prop:powset-is-bool-alg}
Let $X$ be a finite set. By setting the bottom (resp. top) element to be $\emptyset$ (resp. $X$) and meets (resp. joins) to be intersections (resp. unions), $2^X$ forms a finite Boolean algebra.
\end{proposition}

\begin{proof}
The proof follows easily from the properties of elementary set operations. 
\end{proof}

\subsection{Properties of Lattice Morphisms}

For the remainder of this section, $A$ and $B$ are assumed to be finite (hence bounded) lattices. 

\begin{definition}[Lattice Morphism]
$f : A \to B$ is a \textbf{lattice (homo)morphism} if for $x, y \in A$ if $$f(0_A) = 0_B$$ $$f(1_A) = 1_B$$ $$f(x \meet y) = f(x) \meet f(y)$$ $$f(x \join y) = f(x) \join f(y)$$ 
\end{definition}

\begin{lemma}
Lattice morphisms of Boolean algebras preserve complements.
\end{lemma}

\begin{proof}
The proof follows easily from the fact that lattice morphisms preserve top and bottom elements.
\end{proof}

\begin{definition}[Adjoints of Lattice Morphisms]
Let $R : B \to A$. Then $L : A \to B$ is \textbf{left-adjoint} to $R$ (and $R$ is \textbf{right-adjoint} to $L$) if for all $a \in A$ and $b \in B$,

\begin{equation*}
L(a) \leq b \Longleftrightarrow a \leq R(b).
\end{equation*}
\end{definition}

\begin{definition}[Monotone Lattice Morphisms]
A lattice morphism $f : A \to B$ is \textbf{monotone} if for $a, a' \in A$ such that $a \leq a'$ $f(a) \leq f(a')$.
\end{definition}

The following lemmas are also presented without proof since they are nearly identical to the proofs of \hyperref[lem:inv-img-adjoint-lem-1]{Lemmas 1.4-1.8}. The main difference is the use of the partial order $\leq$ instead of $\subseteq$.

\begin{lemma}
If $L$ is left-adjoint to $R$, $a \leq R(L(a))$ and $L(R(b)) \leq b$.
\end{lemma}

\begin{lemma}
If $f : A \to B$ has a left or right adjoint, $f$ is monotone.
\end{lemma}

\begin{lemma}
Let $L : A \to B$ and $R : B \to A$. If $L$ and $R$ are monotone and $a \leq R(L(a))$ and $L(R(b)) \leq b$ for all $a \in A$ and $b \in B$, then $L$ is left-adjoint to $R$.
\end{lemma}

\begin{lemma}
Left and right adjoints (of lattice morphisms) are unique.
\end{lemma}

\begin{proposition}
Recall that $A$ and $B$ are finite lattices. Then any lattice morphism $g : B \to A$ has a left adjoint $g_\exists : A \to B$ and a right adjoint $g_\forall : A \to B$. Moreover, $g_\exists$ and $g_\forall$ preserve joins and meets respectively.
\end{proposition}

\begin{proof}
Like \cref{prop:inv-image-has-adjoints}, we prove more generally that (i) $g$ has a left adjoint $g_\exists$ if and only if $g$ preserves $\meet, 1_B$, and (ii) that $g$ has a right adjoint $g_\forall$ if and only if $g$ preserves $\join, 0_B$. The proof of (ii) is excluded for brevity since it is similar to (i). \\

Suppose $g$ preserves meets. For any $a \in A$, let $S = \{b : a \leq g(b)\}$ and define $g_\exists(a) = \bigmeet_{s \in S}s$. Let $b \leq b'$ for $b, b' \in B$. Since $g(b \meet b') = g(b) \meet g(b') = g(b)$, $g(b) \leq g(b')$, meaning $g$ is monotone. Hence, $g_\exists(a) \leq b \Rightarrow g(b) \geq g(g_\exists(a)) \geq a$ since $g$ preserves meets. The fact that $a \leq g(b) \Rightarrow g_\exists(a) \leq b$ and that $g_\exists$ preserves unions is clear from the definition of $g_\exists$.
\end{proof}

\textbf{Remark:} This generalizes to infinite lattices if we assume that $A$ and $B$ are complete.

\begin{definition}[Completely Join-Prime]
We say that $x \in A \setminus \{0_A\}$ is \textbf{prime} (or \textbf{completely join-prime}) if for all $y, z \in A$, $x \leq y \join z$ implies that $x \leq z$ or $x \leq y.$ Let $\mathcal{J}_A$ denote the set of join-prime elements of $A$. 
\end{definition}

\textbf{Remark:} Similarly, we can say more generally that for a complete lattice $L,$ $x \in L \setminus \{0_L\}$ is completely join-prime if $x \leq \bigjoin_{t \in T} t$ for any $T \subseteq L$ implies that $x \leq t$ for some $t \in T.$


\begin{lemma} \label{lem:left-adjoint-maps-primes-to-primes}
The left adjoint $g_\exists : A \to B$ of a lattice morphism $g : B \to A$ maps prime elements of $A$ to prime elements of $B$.
\end{lemma}

\begin{proof}
Let $a \in A$ be prime, let $b_1, b_2 \in B$ s.t.\ $g_\exists(a) \leq b_1 \join b_2$, and let $t_i = g(b_i)$. Then $a \leq g(b_1 \join b_2) = t_1 \join t_2$ by the definition of the adjoint. Since $a$ is prime and $t_1, t_2 \in B$, $a \leq t_k$ for some $k \in \{1, 2\}$. Because $t_k = g(b_k)$, we have that $g_\exists(a) \leq g_\exists(g(b_k)) \leq b_k$.
\end{proof}

\subsection{Finite Boolean Algebras and Power Sets}

We now turn our attention to Boolean algebras to describe isomorphisms between lattice morphisms of Boolean algebras and maps of sets. Unless specified otherwise, $C, D$ and $X, Y$ will denote finite Boolean algebras and finite sets, respectively, for the remainder of this section. \\

The following lemma shows how primes correspond to atoms (i.e.\ singletons) in the previous section.

\begin{lemma} \label{lem:primes-are-atoms}
$p \in C$ is prime if and only if $z \leq p$ implies $z = 0_C$ or $z = p$. Moreover, for every $x \in C \setminus \{0_C\}$, there exists a prime element $p \in C$ such that $p \leq x$. 
\end{lemma}

\begin{proof}
Suppose $p$ is prime. If $y < p$, $p \leq \neg y$ since $p \leq y \meet \neg y$. Then $y = 0_C$ since $y < p \leq \neg y$ and $y = y \meet \neg y = 0_C$. \\

For the converse, let $a, b \in C$ and suppose $p \leq a \join b$, $p \not\leq a$, and $p \not\leq b$. If $p > a$ or $p > b$, a contradiction is reached since $a = 0_C$ or $a = p$ (and similarly for $b$). Hence $p \join a \neq a$ and $p \join b \neq b$. Then $p \join a > a$ and $p \join b > b$. However, this means that $$p \join a \join p \join b = p \join a \join b > a \join b,$$ a contradiction. \\

Now, for any $x \in C \setminus \{0_C\}$ choose some $y < x$. If $0_C$ is the only such $y$, $x$ is prime and we are done. Setting $x := y$ and repeating this process completes the proof.
\end{proof} 

\begin{proposition} \label{prop:repr-bool-alg-with-primes}
For every $x \in C \setminus \{0_C\}$, there exists exactly one nonempty set $S_x \subseteq \mathcal{J}_C$ such that $x = \bigjoin_{s \in S_x} s$. That is, every element of a finite Boolean algebra can be uniquely represented as a finite join of its prime elements.
\end{proposition}

\begin{proof}
Let $x \in C \setminus \{0_C\}$. Define $S_x = \{p \mid p \leq x : p \text{ is prime}\}$. By \cref{lem:primes-are-atoms}, $S_x$ defined in this way is nonempty. Let $y = \bigjoin_{s \in S_x} s$. If $x \meet \neg y = 0_C$, $x = y$ since complements are unique in finite Boolean algebras by \cref{lem:finite-ba-complements-are-unique}. Suppose $x \meet \neg y \neq 0_C$. Then there exists a prime element $p$ such that $p \leq x \meet \neg y \leq x$. However, this contradicts \cref{lem:de-morgans-laws}, since this means that $p \in S_x$ and $p \leq \neg y$. \\

For uniqueness, suppose there existed another nonempty set of primes $S'_x$ such that $x = \bigjoin_{s \in S'_x} s$. Then $p \leq x$ for every $p \in S'_x$, meaning $S'_x \subseteq S_x$. Suppose $S'_x \neq S$. Then there exists some $p_0 \in S_x$ such that $p_0 \not\in S'_x$ and $\bigjoin_{s \in S'_x} s \join p_0 = \bigjoin_{s \in S'_x} s$. However, this means that $p_0 < s$ for some $s \in S'_x$, a contradiction. Hence $S_x = S'_x$.
\end{proof}

\begin{proposition}
There exists a bijection $C \to 2^X$ for some set $X$. This also shows that the cardinality of every finite Boolean algebra is a power of two. 
\end{proposition}

\begin{proof}
We may let $0_C$ correspond (bijectively) to $\emptyset \in 2^X$ regardless of the choice of $X$. By \cref{prop:repr-bool-alg-with-primes}, for every $x \in C \setminus \{0_C\}$ there exists a unique $S_x$ such that $x = \bigjoin_{s \in S_x} s$. Define $X = \mathcal{J}_C$. Then let each $x \in C \setminus \{0_C\}$ correspond to the set $S_x \in 2^X$ so that $\bigjoin_{s \in S_x} s = x$. Similarly, let each $T \in 2^X$ correspond to $\bigjoin_{t \in T} t$. Thus we have constructed a bijection and the proof is complete.
\end{proof}

\begin{lemma} \label{lem:inv-image-is-lattice-morphism}
Recall that $2^X, 2^Y$ are finite Boolean algebras as defined in \cref{prop:powset-is-bool-alg}. Then the inverse image function of a map $X \to Y$ is a lattice morphism.
\end{lemma}

\begin{proof}
The proof follows from \cref{thm:inv-image-preserves-structure}.
\end{proof}

\begin{theorem}
There exists a bijection between:
\begin{enumerate}
    \item{finite Boolean algebras and finite sets}
    \item{lattice morphisms $D \to C$ and maps $X \to Y$}
\end{enumerate}
(2) provides a result more general to \cref{thm:bijection-set-morphisms-to-powset-morphisms} by removing the assumption that the elements of $C, D$ are themselves sets.
\end{theorem}

\begin{proof}
For (1), we let every Boolean algebra $C$ correspond with the finite set $\mathcal{J}_C$ and let every finite set $X$ correspond with the Boolean algebra formed by $2^X$. \\

For (2), we let every lattice morphism $g : D \to C$ correspond to $\restr{g_\exists}{\mathcal{J}_C}$ (the restriction of the left adjoint to primes) and let every map $f: X \to Y$ correspond to the inverse image function $f^* : 2^Y \to 2^X$, which is a lattice morphism by \cref{lem:inv-image-is-lattice-morphism}. $(f^*)_\exists$ restricted to primes is a unique map $f : \{\{x\} : x \in X\} \to \{\{y\} : y \in Y\}$ and $\restr{g_\exists}{\mathcal{J}_C}$ is a unique map $g : \mathcal{J}_C \to \mathcal{J}_D$ by \cref{lem:left-adjoint-maps-primes-to-primes}. \\

Since $\mathcal{J}_{2^X} = \{\{x\} : x \in X\}$ by \cref{lem:primes-are-atoms} and there exists an obvious bijection $X \to \{\{x\} : x \in X\}$, the proof is complete.
\end{proof}


\end{document}